% !TeX root = ./main.tex

\section{数列极限}
\begin{exposition}
    单调有界数列的收敛性证明需要依靠实数体系的确界定理
\end{exposition}

\begin{exposition}
    在实数体系中,实数的完备性确保了其确界原理。
    确界原理是实数完备性的体现。
    而阿基米德性只是实数完备性的结果,也就是说阿基米德性是实数完备性的必要不充分条件。
\end{exposition}

\begin{exposition}
    数列极限的判别法:
    对于单调数列,若其有界则可以得知其收敛。
    对于一般性数列,若满足柯西准则,则其收敛。
\end{exposition}

\begin{exposition}[上下极限]
    这一概念的提出是为了在不清楚数列特性的情况下,确定其是否收敛。
    这里引出符号 $\overline{a_m} = \sup{\{a_k | k > m, m \in N^+\}}$,
    表示当 $n$ 很大,例如大于 $m$ 时,数列 $\{a_n\}$ 的子集的上确界。
    相当于 $m$ 将数列 $\{a_n\}$ 分成了两类,$\{a_k | k > m\}$ 称之为上类,
    相反为下类。由于我们关注的是 $n \to +\infty$ 时 $\{a_n\}$ 的极限,
    所以我们只取上类。

    与 $\overline{a_m}$ 相对的有 $\underline{a_m} = \inf{\{a_k | k > m, m \in N^+\}}$。
    表示对 $\{a_n\}$ 的上类去下界。

    要注意的时,不管 $\overline{a_m}$ 还是 $\underline{a_m}$ 都是关于 $m$ 的数列。
    当这两个数列足够接近的时候,$\{a_n\}$ 存在极限,或者说收敛。形式化表示为:
    \begin{equation}
        \lim_{n \to +\infty}{a_n} \text{\, exist} \iff 
        \overline{a_n} - \underline{a_n} \to 0 \quad (n \to +\infty)
    \end{equation}
\end{exposition}

\begin{exposition}[柯西准则]
    元素随着序数的增加而愈发的靠近的数列,被称为柯西数列。形式化表示为:
    \begin{equation}
        \lim_{n\to\infty}{(a_i - a_j)} = 0 \quad (i,j > n)
    \end{equation}
    在完备空间中,柯西数列必然收敛,收敛数列也必然时柯西数列。
\end{exposition}

\begin{exposition}[自然常数 $e$]    
    数列 $a_n = (1+\frac{1}{n})^n$ 为单调递增函数。
    数列 $b_n = (1+\frac{1}{n})^{n+1}$ 为单调递减函数。
    \begin{equation}
        \lim_{n \to \infty}{a_n} = \lim_{n \to \infty}{b_n} = e
    \end{equation}
    另外:
    \begin{align*}
        &e_n = 1 + \sum_{i=1}{n}{\frac{1}{i!}} \\
        &a_n < e_n < e \\
        &\lim_{n \to \infty}{e_n} = e
    \end{align*}
\end{exposition}

\begin{exposition}[欧拉常数]
    \begin{align*}
        c_n &= 1 + \frac12 + \frac13 + \cdots + \frac1n - \ln{n}\\
        \gamma &= \lim_{n \to \infty}{c_n} = 0.577215\cdots
    \end{align*}
\end{exposition}

\begin{exposition}[stolz 公式]
    设有集合 $\{a_n\}$、$\{b_n\}$:
    \begin{enumerate}
        \item 若 $\{b_n\}$ 严格单调趋向于 $+\infty$,则
            \begin{align*}
                \lim_{n \to \infty}{\frac{a_n - a_{n-1}}{b_n - b_{n-1}}} &= l \\
                \Rightarrow \lim_{n \to \infty}{\frac{a_n}{b_n}} &= l
            \end{align*}
        \item 若 $\lim \limits_{n \to \infty}{b_n} = 0 \land b_n < b_{n-1} \land \lim_{n \to \infty}{a_n} = 0$,则
            亦可以推出与上面同样的结论。
    \end{enumerate}
\end{exposition}