% !TeX root = ./math-note.tex

\section{函数}

\begin{exposition}
    设有数列 $\{a_n\}$ 那么 $\{f(a_n)\}$ 依旧是数列。
    所以有时候对函数的研究可以转换成对数列的研究。
\end{exposition}

\begin{exposition}
    实数轴上的一点 $x_0$ 等价于所有以该点为极限的数列的并集。
    也就是说所有以 $x_0$ 为极限的数列可以确定实轴上的一个实数点。
\end{exposition}

\begin{note}
    我们从数学分析中应该学到的不只是计算方法,更多的应该是分析问题的方法。
    比如函数关系比较,可以通过无穷小的阶;数列没有最值的时候,可以通过确界属性来进行代替。
\end{note}

\begin{note}
    函数 $f$ 的微分的斜率是 $f$ 的导数。
    微分是关于函数自变量的线性映射(在某点处的切线),而导数则是该点的斜率。
    微分某种意义上给出了函数在极限意义上的一种定义方法(看成 $f$ 的一个线性近似)。
    $x_0$ 处的微分 $df(x_0)$ 可以写成:
    \begin{equation*}
        df(x_0) = f'(x_0)dx(x_0)
    \end{equation*}
    微分的另一种形式,若存在常数 $A$,使得
    \begin{equation*}
        f(x) = f(x_0) + A(x-x_0) + o(x-x_0) \quad (x \to x_0)
    \end{equation*}
    则称 $f$ 在 $x_0$ 处可微,线性映射 $x \mapsto Ax$ 称为 $f$ 在 $x_0$ 处的微分,记为 $df(x_0)$。
    试问 $o(x - x_0)$ 在此处有何意义?
\end{note}

\begin{note}
    导函数的计算由定义可得:
    \begin{equation*}
        f'(x) = \lim_{\delta \to 0}{\frac{f(x+\delta) - f(x)}{\delta}}
    \end{equation*}
\end{note}

\begin{note}
    如果导数表示的瞬时的变化量,那么积分表示一段时间内变化量的累加,也就是总的变化量。
\end{note}

\begin{note}
    我认为,现阶段应当将各个概念理清楚,明白各个定理(概念)之间的脉络关系,每一个定理是如何作用的。
    把握好研究问题的思路(或者说入口点)。
    至于如何计算问题,会在上述完成之后进行专门的训练。
\end{note}

\begin{note}
    函数极限关注的是点,而函数导数关注的是点的邻域。
\end{note}

\begin{note}[微分算子]
    \begin{equation*}
        P\varphi = \sum_{k=0}^{n}{a_k(x)\varphi^{(k)}{(x)}}
    \end{equation*}
    将 $P$ 称之为一个 $n$ 阶微分算子。
    关于未知变量 $\varphi$ 的方程 $P\varphi = f$ 称之为 $n$ 阶微分方程。
    \notetime{2019-11-24 13:39:14}
\end{note}

\begin{note}[Fermat 定理]
    费马定理:
    $x_0$ 为函数 $f$ 在 $I$ 区间内的极值点,且 $x_0$ 为 $I$ 的内点,
    则 $f'(x_0) = 0$。
    
    由 Fermat 定理可以引出达布定理(Darboux) 和 罗尔定理(Rolle)。
    \notetime{2019-11-24 14:10:38}
\end{note}

\begin{note}[Darboux 定理]
    设 $f$ 为 $[a, b]$ 上的可导函数,则 $f'$ 可以取到 $f'_{+}(a)$ 与 $f'_{-}(b)$
    之间的任意值。
    \notetime{2019-11-24 14:16:17}
\end{note}

\begin{note}[Rolle 定理]
    设函数 $f$ 在 $[a, b]$ 上连续,在 $(a,b)$ 中可微,且 $f(a) = f(b)$。
    则存在 $\xi \in (a,b)$,使得 $f'(\xi) = 0$。
    \notetime{2019-11-24 14:20:46}
\end{note}

\begin{note}
    Rolle 定理中为何要求可微,而达布定理中要求可导?这两者之间是否存在联系?
    \notetime{2019-11-24 14:56:34}
\end{note}

\begin{note}[Lagrange]
    Lagrange 定理:设函数 $f$ 在 $[a, b]$ 上连续,则 $(a, b)$ 中可微,则存在 $\xi \in (a, b)$,
    使得
    \begin{equation*}
        f'(\xi) = \frac{f(a) - f(b)}{a - b} \text{, 或}\ f(a) - f(b) = f'(\xi)(a-b).
    \end{equation*}

    Lagrange 定理和微分关系较大一点,表现出函数的线性映射。\strong{(可微和可导一定要分开,两者侧重点不一样)}
    \notetime{2019-11-24 15:02:23}
\end{note}

\begin{note}[Cauchy]
    设函数 $f, g$ 在 $[a, b]$ 上连续,在 $(a, b)$ 中可微,且 $g'(x) \neq 0, \forall x \in (a, b)$。
    则存在 $\xi \in (a, b)$ 使得
    \begin{equation*}
        \frac{f(a) - f(b)}{g(a) - g(b)} = \frac{f'(\xi)}{g'(\xi)}.
    \end{equation*}
    \notetime{2019-11-24 15:24:19}
\end{note}

\begin{note}
    Rolle、Lagrange 和 Cauchy 都属于微分中值定理范畴。
    表述为函数变换量和变化率之间的连续。
    \notetime{2019-11-24 15:26:14}
\end{note}

\begin{note}
    由导数可以引出函数单调性、凹凸性、可逆性等一系列问题。
    同时关注各条件的加强与减弱。
\end{note}